\documentclass{article}

% Citations and Bibtex
\usepackage{cite}
% Because underscores in .bib or something.
% I think underscores in the bibtex name/code is bad.
\usepackage[strings]{underscore}
%\usepackage{url}

% Captions
\usepackage[font=scriptsize]{caption}

% Sections and chapters
%\usepackage{blindtext}
\usepackage[utf8]{inputenc}

% Margins
\usepackage[a4paper,left=3cm,right=2cm,top=2.5cm,bottom=2.5cm]{geometry}

% For tabs
\usepackage{tabto}
\usepackage{parskip}
\setlength{\parindent}{15pt}
\makeatletter
\newcommand\tabfill[1]{%
	\dimen@\linewidth
	\advance\dimen@\@totalleftmargin
	\advance\dimen@-\dimen\@curtab
	\parbox[t]\dimen@{%
		\leftskip=2em\hspace*{-2em}#1\ifhmode\unskip\nobreak\strut\fi
	}%
}
\makeatother

\textwidth=.75\textwidth % just to make wrapping more evident

% For links
\usepackage[hidelinks]{hyperref}
\hypersetup{
	colorlinks,
	citecolor=black,
	filecolor=black,
	linkcolor=black,
	urlcolor=black
}


% For maths
\usepackage{mathtools}
\usepackage{amsmath,amsfonts,amssymb,amsthm, bm}

% For boxes or something
\usepackage{mdframed}

% For frames/boxes
%\usepackage{tcolorbox}

% For C++ code
\usepackage{listings}
\usepackage{lstautogobble}	% For indentation in code (lstlisting)
\usepackage{xcolor}
\usepackage{textcomp}	% Said it needed this for the code
\definecolor{listinggray}{gray}{0.9}
\definecolor{lbcolor}{rgb}{0.9,0.9,0.9}%{1,1,1}
\definecolor{consoleColor}{rgb}{0.7,0.7,0.7}%{1,1,1}
%\definecolor{darkgreen}{rgb}{0.0, 0.9, 0.0}

\lstdefinestyle{cpp}{
	numbers=left,
	backgroundcolor=\color{lbcolor},
	basicstyle=\scriptsize\color{black},
	identifierstyle=\ttfamily,%\color{black},
	keywordstyle=\color[rgb]{0,0,1},
	commentstyle=\color[rgb]{0.0,0.5,0.0},		%% Comment color
	stringstyle=\color[rgb]{0.627,0.126,0.941},
	numberstyle=\color[rgb]{0.205, 0.142, 0.73},
	language=[GNU]C++
}
\lstdefinestyle{console}{
	numbers=none,
	backgroundcolor=\color{consoleColor},
	basicstyle=\scriptsize\color[rgb]{0.0, 0.5, 0.0}
}
\lstdefinestyle{blackAndWhite}{
	numbers=none,
	backgroundcolor=\color{white},
	basicstyle=\scriptsize\color{black}
}

\lstset{
	autogobble=true,	% Added
%	backgroundcolor=\color{lbcolor},
	tabsize=4,    
%   rulecolor=,
%	language=[GNU]C++,
	basicstyle=\scriptsize,
	upquote=true,
%	aboveskip={1.5\baselineskip},
	aboveskip=\medskipamount,	% Changed
	columns=fixed,
	showstringspaces=false,
	extendedchars=false,
	breaklines=true,
	prebreak = \raisebox{0ex}[0ex][0ex]{\ensuremath{\hookleftarrow}},
	frame=single,
	rulecolor=\color{black},	% Frame color
%	numbers=left,
	showtabs=false,
	showspaces=false,
	showstringspaces=false,
%	identifierstyle=\ttfamily,
%	keywordstyle=\color[rgb]{0,0,1},
%	commentstyle=\color[rgb]{0.0,0.5,0.0},		%% Comment color
%	stringstyle=\color[rgb]{0.627,0.126,0.941},
%	numberstyle=\color[rgb]{0.205, 0.142, 0.73}
	%\lstdefinestyle{C++}{language=C++,style=numbers}’.
}




\begin{document}
	% Make title
	\title{C++ Assignment - Matrix Maths}
	\date{\today}
	\author{Paul Knutson}
	\maketitle
	\thispagestyle{empty}
	
	% Horizontal line
	\begin{center}
		\line(1,0){350}
	%\end{center}
	
	% for prettificatory reasons
	\hfill \break
	% Abstract
	The assignment is to create a matrix multiplier program. It consists of three elements, where each can be cumulatively added to the program. It is to be coded in C++.
	
	% Horizontal line
	%\begin{center}
		\line(1,0){350}
	\end{center}
	
	\clearpage

	\tableofcontents{}
	\clearpage
	
	
	
	\section{Introduction}
		\subsection{Part 1}
			The first part of the C++ program for matrix multiplication should have the following specifications:
			\begin{itemize}
				\item Use constructor dynamic memory allocation for the matrix.
				\item Use a getData() function to input values for the matrix.
				\item Use show() to display the matrix.
				\item Use mul() to multiply two matrices.
			\end{itemize}
		
			Example console output:
			\begin{lstlisting}[style=console]
				enter m,n:2 2
				
				Enter the matrix of size 2x2:
				2 3
				2 3
				
				Enter the matrix of size 2x2:
				4 5
				4 5
				
				The matrix for object p:
				2 3
				2 3
				
				The matrix for object q:
				4 5
				4 5
				
				Enter resultant matrix:
				20 25
				20 25
			\end{lstlisting}
		% End of Part 1
		
		\subsection{Part 2}
			The first part of the C++ program for matrix multiplication should have the following specifications:
			\begin{itemize}
				\item Use operator*() for matrix multiplication instead of mul()
				\item Make operator*() as a friend function
			\end{itemize}
		% End of Part 2
		
		\subsection{Part 3}
			The program should perform the write operatorion to a file in addition to screen.
			
			Output:
			\begin{lstlisting}[style=console]
				Enter file name> one.txt
				Enter contents to store in file
			\end{lstlisting}
			
		% End of Part 3
	% End of section 1 - Introduction
	
	
	\section{Method}
		\subsection{Matrix multiplication}
		A matrix multiplication between two matrices $A$ and $B$ of size $m \times t$ and $t \times n$ will result in a matrix of size $m \times p$, and $t$ number of addends per index of the matrix. \\\\
			%\tiny \\
			%\begin{figure}
				\tiny
				\fbox{\begin{minipage}{\textwidth}
					$
						X = 
						\begin{bmatrix}
							a & b \\
							c & d
						\end{bmatrix}
						, Y =
						\begin{bmatrix}
							e & f \\
							g & h
						\end{bmatrix}
						, XY = 
						\begin{bmatrix}
							ae + bg & af + bh \\
							ce + dg & cf + gh
						\end{bmatrix}
					$
				\end{minipage}}
				\captionof{figure}{Simple matrix multiplication example}
				%\caption{Simple matrix multiplication example}
			%\end{figure}
			\hfill \break \\\\
			%\begin{figure}
				\tiny
				\fbox{\begin{minipage}{\textwidth}
					$
						Q = 
						\begin{bmatrix}
							X_{00} & X_{01} \\
							X_{10} & X_{11}
						\end{bmatrix}
						, P =
						\begin{bmatrix}
							Y_{00} & Y_{01} \\
							Y_{10} & Y_{11}
						\end{bmatrix}
						, QP = 
						\begin{bmatrix}
							X_{00}Y_{00} + X_{01}Y_{10} & X_{00}Y_{01} + X_{01}Y_{11} \\
							X_{10}Y_{00} + X_{11}Y_{10} & X_{10}Y_{01} + X_{11}Y_{11}
						\end{bmatrix}
					$
				\end{minipage}}
				\captionof{figure}{Simple matrix multiplication indexed example}
				%\caption{Simple matrix multiplication example}
			%\end{figure} \\
			\hfill \break \\\\
			%\begin{figure}
				\tiny
				\fbox{\begin{minipage}{\textwidth}
					%\begin{equation}
					$
						A: 2 \times 3, B: 3 \times 4, AB: 2 \times 4 \\
						\begin{aligned}
							A &= 
							\begin{bmatrix}
								A_{00} & A_{01} & A_{02} \\
								A_{10} & A_{11} & A_{12}
							\end{bmatrix}
							, B =
							\begin{bmatrix}
								B_{00} & B_{01} & B_{02} & B_{03} \\
								B_{10} & B_{11} & B_{12} & B_{13} \\
								B_{20} & B_{21} & B_{22} & B_{23}
							\end{bmatrix} \\
							AB &= 
							\begin{bmatrix}
								A_{00}B_{00} + A_{01}B_{10} & A_{00}B_{01} + A_{01}B_{11} & A_{00}B_{02} + A_{01}B_{12} \\
								A_{10}B_{00} + A_{01}B_{10} & A_{10}B_{01} + A_{01}B_{11} & A_{10}B_{02} + A_{01}B_{12} \\
								A_{20}B_{00} + A_{01}B_{10} & A_{20}B_{01} + A_{01}B_{11} & A_{20}B_{02} + A_{01}B_{12}
							\end{bmatrix} \\
							BA &= 
							\begin{bmatrix}
								B_{00}A_{00} + B_{01}A_{10} + B_{02}A_{20} & B_{00}A_{01} + B_{01}A_{11} + B_{02}A_{21} \\
								B_{10}A_{00} + B_{01}A_{10} + B_{02}A_{20} & B_{10}A_{01} + B_{01}A_{11} + B_{02}A_{21} \\
								B_{20}A_{00} + B_{01}A_{10} + B_{02}A_{20} & B_{20}A_{01} + B_{01}A_{11} + B_{02}A_{21}
							\end{bmatrix} \\
							A &= 
							\begin{bmatrix}
								1 & 2 & 3 \\
								4 & 5 & 6
							\end{bmatrix}
							, B =
							\begin{bmatrix}
								1 & 2 & 3 & 4 \\
								5 & 6 & 7 & 8 \\
								9 & 10 & 11 & 12
							\end{bmatrix} \\
							AB &= 
							\begin{bmatrix}
								1*1 + 2*5 + 3*9 &
								1*2 + 2*6 + 3*10 &
								1*3 + 2*7 + 3*11 &
								1*4 + 2*8 + 3*12 \\
								
								4*1 + 5*5 + 6*9 &
								4*2 + 5*6 + 6*10 &
								4*3 + 5*7 + 6*11 &
								4*4 + 5*8 + 6*12
							\end{bmatrix} \\
							&= 
							\begin{bmatrix}
								1 + 10 + 27 &
								2 + 12 + 30 &
								3 + 14 + 33 &
								4 + 16 + 36 \\
								
								4 + 25 + 54 &
								8 + 30 + 60 &
								12 + 35 + 66 &
								16 + 40 + 72
							\end{bmatrix}
							= 
							\begin{bmatrix}
								38 &
								54 &
								50 &
								56 \\
								
								83 &
								98 &
								113 &
								128
							\end{bmatrix}
						\end{aligned}
					$
					%\end{equation}
				\end{minipage}}
				\captionof{figure}{Full example of matrix multiplication}
				%\caption{Simple matrix multiplication example}
			%\end{figure} \\
			\normalsize
			\hfill \break \\\\
		% End of subsection - Matrix multiplication
	% End of section 2 - Method
	
	
	\section{Result}
		
	% End of section 3 - Results
	
	
	\section{Discussion}
		
	% End of section 4 - Discussion
	
	% Horizontal line
	\begin{center}
		\line(1,0){350}
	\end{center}
	-
	\\ \\ \\ \\ \\ \\
	GitHub repository for the project: \\
	\href{https://github.com/catsymptote/matrixMaths}{Web repo} \\
	\href{https://github.com/catsymptote/matrixMaths.git}{.git}
	
	
	\clearpage
	
	\bibliographystyle{plain}
	\bibliography{biblib}
	
	\listoftables
	\listoffigures
\end{document}
