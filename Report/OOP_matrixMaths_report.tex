\documentclass{article}

% Citations and Bibtex
\usepackage{cite}
% Because underscores in .bib or something.
% I think underscores in the bibtex name/code is bad.
	\usepackage[strings]{underscore}
%\usepackage{url}

% Sections and chapters
%\usepackage{blindtext}
\usepackage[utf8]{inputenc}

% Margins
\usepackage[a4paper,left=3cm,right=2cm,top=2.5cm,bottom=2.5cm]{geometry}

% For tabs
\usepackage{tabto}
\usepackage{parskip}
\setlength{\parindent}{15pt}
\makeatletter
\newcommand\tabfill[1]{%
	\dimen@\linewidth
	\advance\dimen@\@totalleftmargin
	\advance\dimen@-\dimen\@curtab
	\parbox[t]\dimen@{%
		\leftskip=2em\hspace*{-2em}#1\ifhmode\unskip\nobreak\strut\fi
	}%
}
\makeatother

\textwidth=.75\textwidth % just to make wrapping more evident

% For links
\usepackage[hidelinks]{hyperref}
\hypersetup{
	colorlinks,
	citecolor=black,
	filecolor=black,
	linkcolor=black,
	urlcolor=black
}


% For maths
\usepackage{mathtools}


% For C++ code
\usepackage{listings}
\usepackage{lstautogobble}	% For indentation in code (lstlisting)
\usepackage{xcolor}
\usepackage{textcomp}	% Said it needed this for the code
\definecolor{listinggray}{gray}{0.9}
\definecolor{lbcolor}{rgb}{0.9,0.9,0.9}%{1,1,1}
\definecolor{consoleColor}{rgb}{0.7,0.7,0.7}%{1,1,1}
%\definecolor{darkgreen}{rgb}{0.0, 0.9, 0.0}

\lstdefinestyle{cpp}{
	numbers=left,
	backgroundcolor=\color{lbcolor},
	basicstyle=\scriptsize\color{black},
	identifierstyle=\ttfamily,%\color{black},
	keywordstyle=\color[rgb]{0,0,1},
	commentstyle=\color[rgb]{0.0,0.5,0.0},		%% Comment color
	stringstyle=\color[rgb]{0.627,0.126,0.941},
	numberstyle=\color[rgb]{0.205, 0.142, 0.73},
	language=[GNU]C++
}
\lstdefinestyle{console}{
	numbers=none,
	backgroundcolor=\color{consoleColor},
	basicstyle=\scriptsize\color[rgb]{0.0, 0.5, 0.0}
}
\lstdefinestyle{blackAndWhite}{
	numbers=none,
	backgroundcolor=\color{white},
	basicstyle=\scriptsize\color{black}
}

\lstset{
	autogobble=true,	% Added
%	backgroundcolor=\color{lbcolor},
	tabsize=4,    
%   rulecolor=,
%	language=[GNU]C++,
	basicstyle=\scriptsize,
	upquote=true,
%	aboveskip={1.5\baselineskip},
	aboveskip=\medskipamount,	% Changed
	columns=fixed,
	showstringspaces=false,
	extendedchars=false,
	breaklines=true,
	prebreak = \raisebox{0ex}[0ex][0ex]{\ensuremath{\hookleftarrow}},
	frame=single,
	rulecolor=\color{black},	% Frame color
%	numbers=left,
	showtabs=false,
	showspaces=false,
	showstringspaces=false,
%	identifierstyle=\ttfamily,
%	keywordstyle=\color[rgb]{0,0,1},
%	commentstyle=\color[rgb]{0.0,0.5,0.0},		%% Comment color
%	stringstyle=\color[rgb]{0.627,0.126,0.941},
%	numberstyle=\color[rgb]{0.205, 0.142, 0.73}
	%\lstdefinestyle{C++}{language=C++,style=numbers}’.
}




\begin{document}
	% Make title
	\title{Introductory Assignment - C++ Calculator}
	\date{\today}
	\author{Paul Knutson}
	\maketitle
	\thispagestyle{empty}
	
	% Horizontal line
	\begin{center}
		\line(1,0){350}
	\end{center}
	
	% for prettificatory reasons
	\hfill \break
	% Abstract
	
	
	% Horizontal line
	\begin{center}
		\line(1,0){350}
	\end{center}
	
	\clearpage

	\tableofcontents{}
	\clearpage
	
	
	
	\section{Introduction}
		\subsection{Part 1}
			The first part of the C++ program for matrix multiplication should have the following specifications:
			\begin{itemize}
				\item Use constructor dynamic memory allocation for the matrix.
				\item Use a getData() function to input values for the matrix.
				\item Use show() to display the matrix.
				\item Use mul() to multiply two matricies.
			\end{itemize}
		
			Example console output:
			\begin{lstlisting}[style=console]
				enter m,n:2 2
				
				Enter the matrix of size 2x2:
				2 3
				2 3
				
				Enter the matrix of size 2x2:
				4 5
				4 5
				
				The matrix for object p:
				2 3
				2 3
				
				The matrix for object q:
				4 5
				4 5
				
				Enter resultant matrix:
				20 25
				20 25
			\end{lstlisting}
		% End of Part 1
		
		\subsection{Part 2}
			The first part of the C++ program for matrix multiplication should have the following specifications:
			\begin{itemize}
				\item Use operator*() for matrix multiplication instead of mul()
				\item Make operator*() as a friend function
			\end{itemize}
		% End of Part 2
		
		\subsection{Part 3}
			The program should perform the write operatorion to a file in addition to screen.
			
			Output:
			\begin{lstlisting}[style=console]
				Enter file name> one.txt
				Enter contents to store in file
			\end{lstlisting}
			
		% End of Part 3
	% End of section 1 - Introduction
	
	
	\section{Method}
		\subsection{Matrix multiplication}
			\tiny
			$
				\begin{bmatrix}
					A_{0,0}	& A_{0, 1}	& A_{0,2} \\\\
					A_{1,0}	& A_{1, 1}	& A_{1,2} \\\\
					A_{2,0}	& A_{2, 1}	& A_{2,2}
				\end{bmatrix}
				\begin{bmatrix}
					B_{0,0}	& B_{0, 1}	& B_{0,2} \\\\
					B_{1,0}	& B_{1, 1}	& B_{1,2} \\\\
					B_{2,0}	& B_{2, 1}	& B_{2,2}
				\end{bmatrix}
			$ \\\\
			$
				=
				\begin{bmatrix}
					A_{0,0}B_{0,0} + A_{0,1}B_{1,0} + A_{0,2}B_{2,0} &
					A_{0,0}B_{0,1} + A_{0,1}B_{1,1} + A_{0,2}B_{2,1} &
					A_{0,0}B_{0,2} + A_{0,1}B_{1,2} + A_{0,2}B_{2,2} \\\\
					
					A_{1,0}B_{0,0} + A_{1,1}B_{1,0} + A_{1,2}B_{2,0} &
					A_{1,0}B_{0,1} + A_{1,1}B_{1,1} + A_{1,2}B_{2,1} &
					A_{1,0}B_{0,2} + A_{1,1}B_{1,2} + A_{1,2}B_{2,2} \\\\
					
					A_{2,0}B_{0,0} + A_{2,1}B_{1,0} + A_{2,2}B_{2,0} &
					A_{2,0}B_{0,1} + A_{2,1}B_{1,1} + A_{2,2}B_{2,1} &
					A_{2,0}B_{0,2} + A_{2,1}B_{1,2} + A_{2,2}B_{2,2}
				\end{bmatrix}
			$\\\\\\
			$
				A = 
				\begin{bmatrix}
					A_{00} & A_{01} \\
					A_{10} & A_{11} \\
					A_{20} & A_{21}
				\end{bmatrix}
				,
				B = 
				\begin{bmatrix}
					B_{00} & B_{01} \\
					B_{10} & B_{11} \\
					B_{20} & B_{21}
				\end{bmatrix}
			$ \\\\
			$
				AB = 
				\begin{bmatrix}
					A_{00}B_{00} + A_{01}B_{10} & A_{00}B_{01} + A_{01}B_{11} & A_{00}B_{02} + A_{01}B_{12} \\
					A_{10}B_{00} + A_{01}B_{10} & A_{10}B_{01} + A_{01}B_{11} & A_{10}B_{02} + A_{01}B_{12} \\
					A_{20}B_{00} + A_{01}B_{10} & A_{20}B_{01} + A_{01}B_{11} & A_{20}B_{02} + A_{01}B_{12}
				\end{bmatrix}
			$\\\\
			$	
				BA = 
				\begin{bmatrix}
					B_{00}A_{00} + B_{01}A_{10} + B_{02}A_{20} & B_{00}A_{01} + B_{01}A_{11} + B_{02}A_{21} \\
					B_{10}A_{00} + B_{01}A_{10} + B_{02}A_{20} & B_{10}A_{01} + B_{01}A_{11} + B_{02}A_{21} \\
					B_{20}A_{00} + B_{01}A_{10} + B_{02}A_{20} & B_{20}A_{01} + B_{01}A_{11} + B_{02}A_{21}
				\end{bmatrix}
			$
			\normalsize
		% End of subsection - Matrix multiplication
	% End of section 2 - Method
	
	
	\section{Result}
		
	% End of section 3 - Results
	
	
	\section{Discussion}
		
	% End of section 4 - Discussion
	
	% Horizontal line
	\begin{center}
		\line(1,0){350}
	\end{center}
	-
	\\ \\ \\ \\ \\ \\
	GitHub repository for the project: \\
	\href{https://github.com/catsymptote/matrixMaths}{Web repo} \\
	\href{https://github.com/catsymptote/matrixMaths.git}{.git}
	
	
	\clearpage
	
	\bibliographystyle{plain}
	\bibliography{biblib}
\end{document}
