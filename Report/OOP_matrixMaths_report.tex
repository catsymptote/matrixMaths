\documentclass{article}

% Citations and Bibtex
\usepackage{cite}
% Because underscores in .bib or something.
% I think underscores in the bibtex name/code is bad.
\usepackage[strings]{underscore}
%\usepackage{url}

% Captions
\usepackage[font=scriptsize]{caption}

% Sections and chapters
%\usepackage{blindtext}
\usepackage[utf8]{inputenc}

% Margins
\usepackage[a4paper,left=3cm,right=2cm,top=2.5cm,bottom=2.5cm]{geometry}

% For tabs
\usepackage{tabto}
\usepackage{parskip}
\setlength{\parindent}{15pt}
\makeatletter
\newcommand\tabfill[1]{%
	\dimen@\linewidth
	\advance\dimen@\@totalleftmargin
	\advance\dimen@-\dimen\@curtab
	\parbox[t]\dimen@{%
		\leftskip=2em\hspace*{-2em}#1\ifhmode\unskip\nobreak\strut\fi
	}%
}
\makeatother

\textwidth=.75\textwidth % just to make wrapping more evident

% For links
\usepackage[hidelinks]{hyperref}
\hypersetup{
	colorlinks,
	citecolor=black,
	filecolor=black,
	linkcolor=black,
	urlcolor=black
}


% For maths
\usepackage{mathtools}
\usepackage{amsmath,amsfonts,amssymb,amsthm, bm}

% For boxes or something
\usepackage{mdframed}

% For frames/boxes
%\usepackage{tcolorbox}

% For C++ code
\usepackage{listings}
\usepackage{lstautogobble}	% For indentation in code (lstlisting)
\usepackage{xcolor}
\usepackage{textcomp}	% Said it needed this for the code
\definecolor{listinggray}{gray}{0.9}
\definecolor{lbcolor}{rgb}{0.9,0.9,0.9}%{1,1,1}
\definecolor{consoleColor}{rgb}{0.7,0.7,0.7}%{1,1,1}
%\definecolor{darkgreen}{rgb}{0.0, 0.9, 0.0}

\lstdefinestyle{cpp}{
	numbers=left,
	backgroundcolor=\color{lbcolor},
	basicstyle=\scriptsize\color{black},
	identifierstyle=\ttfamily,%\color{black},
	keywordstyle=\color[rgb]{0,0,1},
	commentstyle=\color[rgb]{0.0,0.5,0.0},		%% Comment color
	stringstyle=\color[rgb]{0.627,0.126,0.941},
	numberstyle=\color[rgb]{0.205, 0.142, 0.73},
	language=[GNU]C++
}
\lstdefinestyle{console}{
	numbers=none,
	backgroundcolor=\color{consoleColor},
	basicstyle=\scriptsize\color[rgb]{0.0, 0.5, 0.0}
}
\lstdefinestyle{blackAndWhite}{
	numbers=none,
	backgroundcolor=\color{white},
	basicstyle=\scriptsize\color{black}
}

\lstset{
	autogobble=true,	% Added
%	backgroundcolor=\color{lbcolor},
	tabsize=4,    
%   rulecolor=,
%	language=[GNU]C++,
	basicstyle=\scriptsize,
	upquote=true,
%	aboveskip={1.5\baselineskip},
	aboveskip=\medskipamount,	% Changed
	columns=fixed,
	showstringspaces=false,
	extendedchars=false,
	breaklines=true,
	prebreak = \raisebox{0ex}[0ex][0ex]{\ensuremath{\hookleftarrow}},
	frame=single,
	rulecolor=\color{black},	% Frame color
%	numbers=left,
	showtabs=false,
	showspaces=false,
	showstringspaces=false,
%	identifierstyle=\ttfamily,
%	keywordstyle=\color[rgb]{0,0,1},
%	commentstyle=\color[rgb]{0.0,0.5,0.0},		%% Comment color
%	stringstyle=\color[rgb]{0.627,0.126,0.941},
%	numberstyle=\color[rgb]{0.205, 0.142, 0.73}
	%\lstdefinestyle{C++}{language=C++,style=numbers}’.
}




\begin{document}
	% Make title
	\title{C++ Assignment - Matrix Maths}
	\date{\today}
	\author{Paul Knutson}
	\maketitle
	\thispagestyle{empty}
	
	% Horizontal line
	\begin{center}
		\line(1,0){350}
	%\end{center}
	
	% for prettificatory reasons
	\hfill \break
	% Abstract
	This is the report for the second C++ assignment in the HSN/USN Masters in System Engineering with Embedded Systems 2019 - Object Oriented Embedded Systems Programming Languages, fall 2017.
	
	The assignment is to create a matrix multiplier program. It consists of a few incremental improvements, where each can be cumulatively added to the program. It is to be coded in C++.
	
	% Horizontal line
	%\begin{center}
		\line(1,0){350}
	\end{center}
	
	\clearpage

	\tableofcontents{}
	\clearpage
	
	
	
	\section{Introduction}
		\subsection{Part 1}
			The first part of the C++ program for matrix multiplication should have the following specifications:
			\begin{itemize}
				\item Use constructor dynamic memory allocation for the matrix.
				\item Use a getData() function to input values for the matrix.
				\item Use show() to display the matrix.
				\item Use mul() to multiply two matrices.
			\end{itemize}
		
			Example console output:
			\begin{lstlisting}[style=console]
				enter m,n:2 2
				
				Enter the matrix of size 2x2:
				2 3
				2 3
				
				Enter the matrix of size 2x2:
				4 5
				4 5
				
				The matrix for object p:
				2 3
				2 3
				
				The matrix for object q:
				4 5
				4 5
				
				Enter resultant matrix:
				20 25
				20 25
			\end{lstlisting}
		% End of Part 1
		
		\subsection{Part 2}
			The first part of the C++ program for matrix multiplication should have the following specifications:
			\begin{itemize}
				\item Use operator*() for matrix multiplication instead of mul()
				\item Make operator*() as a friend function
			\end{itemize}
		% End of Part 2
		
		\subsection{Part 3}
			The program should perform the write operatorion to a file in addition to screen.
			
			Output:
			\begin{lstlisting}[style=console]
				Enter file name> one.txt
				Enter contents to store in file
			\end{lstlisting}
			
		% End of Part 3
	% End of section 1 - Introduction
	
	
	\section{Method}
		\subsection{Matrix multiplication}
		A matrix multiplication between two matrices $A$ and $B$ of size $m \times t$ and $t \times n$ will result in a matrix of size $m \times p$, and $t$ number of addends per index of the matrix. \\\\
			%\tiny \\
			%\begin{figure}
				\tiny
				\fbox{\begin{minipage}{\textwidth}
					$
						X = 
						\begin{bmatrix}
							a & b \\
							c & d
						\end{bmatrix}
						, Y =
						\begin{bmatrix}
							e & f \\
							g & h
						\end{bmatrix}
						, XY = 
						\begin{bmatrix}
							ae + bg & af + bh \\
							ce + dg & cf + gh
						\end{bmatrix}
					$
				\end{minipage}}
				\captionof{figure}{Simple matrix multiplication example}
				%\caption{Simple matrix multiplication example}
			%\end{figure}
			\hfill \break \\\\
			%\begin{figure}
				\tiny
				\fbox{\begin{minipage}{\textwidth}
					$
						Q = 
						\begin{bmatrix}
							X_{00} & X_{01} \\
							X_{10} & X_{11}
						\end{bmatrix}
						, P =
						\begin{bmatrix}
							Y_{00} & Y_{01} \\
							Y_{10} & Y_{11}
						\end{bmatrix}
						, QP = 
						\begin{bmatrix}
							X_{00}Y_{00} + X_{01}Y_{10} & X_{00}Y_{01} + X_{01}Y_{11} \\
							X_{10}Y_{00} + X_{11}Y_{10} & X_{10}Y_{01} + X_{11}Y_{11}
						\end{bmatrix}
					$
				\end{minipage}}
				\captionof{figure}{Simple matrix multiplication indexed example}
				%\caption{Simple matrix multiplication example}
			%\end{figure} \\
			\hfill \break \\\\
			%\begin{figure}
				\tiny
				\fbox{\begin{minipage}{\textwidth}
					%\begin{equation}
					$
						A: 2 \times 3, B: 3 \times 4, AB: 2 \times 4 \\
						\begin{aligned}
							A &= 
							\begin{bmatrix}
								A_{00} & A_{01} & A_{02} \\
								A_{10} & A_{11} & A_{12}
							\end{bmatrix}
							, B =
							\begin{bmatrix}
								B_{00} & B_{01} & B_{02} & B_{03} \\
								B_{10} & B_{11} & B_{12} & B_{13} \\
								B_{20} & B_{21} & B_{22} & B_{23}
							\end{bmatrix} \\
							AB &= 
							\begin{bmatrix}
								A_{00}B_{00} + A_{01}B_{10} & A_{00}B_{01} + A_{01}B_{11} & A_{00}B_{02} + A_{01}B_{12} \\
								A_{10}B_{00} + A_{01}B_{10} & A_{10}B_{01} + A_{01}B_{11} & A_{10}B_{02} + A_{01}B_{12} \\
								A_{20}B_{00} + A_{01}B_{10} & A_{20}B_{01} + A_{01}B_{11} & A_{20}B_{02} + A_{01}B_{12}
							\end{bmatrix} \\
							BA &= 
							\begin{bmatrix}
								B_{00}A_{00} + B_{01}A_{10} + B_{02}A_{20} & B_{00}A_{01} + B_{01}A_{11} + B_{02}A_{21} \\
								B_{10}A_{00} + B_{01}A_{10} + B_{02}A_{20} & B_{10}A_{01} + B_{01}A_{11} + B_{02}A_{21} \\
								B_{20}A_{00} + B_{01}A_{10} + B_{02}A_{20} & B_{20}A_{01} + B_{01}A_{11} + B_{02}A_{21}
							\end{bmatrix} \\
							A &= 
							\begin{bmatrix}
								1 & 2 & 3 \\
								4 & 5 & 6
							\end{bmatrix}
							, B =
							\begin{bmatrix}
								1 & 2 & 3 & 4 \\
								5 & 6 & 7 & 8 \\
								9 & 10 & 11 & 12
							\end{bmatrix} \\
							AB &= 
							\begin{bmatrix}
								1*1 + 2*5 + 3*9 &
								1*2 + 2*6 + 3*10 &
								1*3 + 2*7 + 3*11 &
								1*4 + 2*8 + 3*12 \\
								
								4*1 + 5*5 + 6*9 &
								4*2 + 5*6 + 6*10 &
								4*3 + 5*7 + 6*11 &
								4*4 + 5*8 + 6*12
							\end{bmatrix} \\
							&= 
							\begin{bmatrix}
								1 + 10 + 27 &
								2 + 12 + 30 &
								3 + 14 + 33 &
								4 + 16 + 36 \\
								
								4 + 25 + 54 &
								8 + 30 + 60 &
								12 + 35 + 66 &
								16 + 40 + 72
							\end{bmatrix}
							= 
							\begin{bmatrix}
								38 &
								54 &
								50 &
								56 \\
								
								83 &
								98 &
								113 &
								128
							\end{bmatrix}
						\end{aligned}
					$
					%\end{equation}
				\end{minipage}}
				\captionof{figure}{Full example of matrix multiplication}
				%\caption{Simple matrix multiplication example}
			%\end{figure} \\
			\normalsize
			\hfill \break \\\\
		% End of subsection - Matrix multiplication
		\subsection{Versions}
			As with the previous assignment
			[see attachment 2]
			I wanted to fulfil the requirements incrementally, each with a new functional version.
			I made it into two versions, where the first version upholds most of the requirements, but lacks proper memory allocation (creating objects on the heap, instead of the stack) and operator overloading.
			The second version creates both the matrices and the arrays on the heap (using the "new" keyword).
			It also includes operator overloading functions for multiplication (*), addition (+), and subtraction (-).
			These are friend functions of the class, which means they are not members of the class, but has access to its private and protected members.
			\cite{cpplang4}.
			The second version also has added the extra test feature of automatically generating an $8 \times 8$ matrix and running it through the operators.
		% End of subsection - Versions
		
		\subsection{Final code}
			% v1 is WITHOUT memory heap stuff and operator overloading.
			% v2 is WITH memory heap stuff and operator overloading.
			
			The includes the program are using:
			\begin{lstlisting}[style=cpp]
				#include <iostream> /// In/out stream
				#include <fstream>  /// File stream
				#include <sstream>  /// String stream
			\end{lstlisting}

			
			Below are the header file, which includes the initialization of the class members and the friend functions. There are a couple of different constructors included.
			You can find the assignment specified getData() which gets data input from the user and show() which outputs the matrix to the console for the user to see.
			There are also two different public print methods. printFile() prints the matrix to the file in a matrix-like appearance, while printFileWA() prints the mathematics in a format made for easy copy/pasting to			\href{https://www.wolframalpha.com/}{Wolfram Alpha}
			which is used to check if the result is correct.
			
			With the private members we can find the sizes of the matrix (m and n) and the matrix itself (mx). mx is a 2D array on the heap, and is created using the make() method.
			The mathematical methods can be found here, as well as two methods to check if the mathematical operator ran is in fact a legal one.
			
			Finally, we have the friendly neighbour operator overloaders for +, - and * outside the namespace keywords.
			\begin{lstlisting}[style=cpp]
				class matrix
				{
					public:
						/// Constructors and destructor.
						matrix();
						matrix(int m, int n);
						matrix(double **mx, int m, int n);
						virtual ~matrix();
						
						void getData();                     /// Get matrix data from user.
						void show(std::string txt = "");    /// Display the matrix to the user.
						
						void printFile(std::string name);   /// Print matrix to file.
						void printFileWA(std::string name); /// Print Wolfram Alpha format to file.
					
					private:
						int m, n;       /// Matrix dimensions
						double** mx;    /// Matrix pointers
						
						matrix mul(matrix &X, matrix &Y);   /// Matrix multiplication
						matrix add(matrix &X, matrix &Y);   /// Matrix addition
						matrix subt(matrix &X, matrix &Y); 	/// Matrix subtraction
						
						bool bMul(matrix &X, matrix &Y);    /// Matrix multiplication legality checker.
						bool bAdd(matrix &X, matrix &Y);  	/// Matrix addition legality checker.
						
						void printToFile(std::string text, std::string filename);   /// Create file from string.
						
						void make(int m, int n);    /// Make the (local) matrix based on size inputs.
					
					/// (Friendly) operator overloading.
					friend matrix operator+(matrix &X, matrix &Y);
					friend matrix operator-(matrix &X, matrix &Y);
					friend matrix operator*(matrix &X, matrix &Y);
				};
			\end{lstlisting}
			The rest of the source code can be found in the git repository on the github link (see bottom of document), or in the attachments (attachment 1).
		% End of subsection - Version 2
	% End of section 2 - Method
	
	
	\section{Result}
		The resulting program can add, subtract and multiply matrices specified by the user, as well as test matrices specified in the source code.
		The user can input the code by first typing the size (m n) of the first matrix, then input row by row like the following:
		\begin{lstlisting}[style=console]
			Input matrix dimensions: m n> 2 2
			1 2
			3 4
			Input matrix dimensions: m n> 2 2
			5 6
			7 8
		\end{lstlisting}
		Note that the same information can be filled in in a single line like the following:
		\begin{lstlisting}[style=console]
			Input matrix dimensions: m n> 2 2 1 2 3 4 2 2 5 6 7 8
			Input matrix dimensions: m n>
		\end{lstlisting}
		This will then output the results from the different mathematical operations using the two matrices:
		\begin{lstlisting}[style=console]
			A
			---------
			1       2
			
			3       4
			---------
			
			B
			---------
			5       6
			
			7       8
			---------
			
			AB
			----------
			19      22
			
			43      50
			----------
			
			A+B
			----------
			6       8
			
			10      12
			----------
			
			A-B
			----------
			-4      -4
			
			-4      -4
			----------
		\end{lstlisting}
		The program also outputs the automatically generated test matrices X and Y and their results.
	% End of section 3 - Results
	
	
	\section{Discussion}
		[the maths itself takes a lot less time than the rest of the programming]
		One of the things I noticed was rather different with this assignment compared to the first one
		[see attachment 2]
		was that the mathematical and numerical logic was far less of the total work on this task.
		What took most of the time was the programming-related aspects of the code.
	% End of section 4 - Discussion
	
	% Horizontal line
	\begin{center}
		\line(1,0){350}
	\end{center}
	-
	\\ \\ \\ \\ \\ \\
	GitHub repository for the project: \\
	\href{https://github.com/catsymptote/matrixMaths}{Web repo} \\
	\href{https://github.com/catsymptote/matrixMaths.git}{.git}
	
	
	
	
	
	\clearpage
	
	
	\setcounter{secnumdepth}{0} %% no numbering
	\hfill \break
	\section{Attachments}
	\begin{enumerate}
		\item Source code for this project - "matrixMaths.zip"
		\item Report for previous assignment - "calculator_report.pdf"
	\end{enumerate}
	
	
	\bibliographystyle{plain}
	\bibliography{biblib}
	
	%\listoftables
	\listoffigures
		
\end{document}
